In the current context of rapid global technological development, cutting-edge technologies such as the Internet of Things and artificial intelligence have created an urgent demand for high-efficiency, low-energy consumption chips that can be quickly constructed. This demand has driven significant innovations in chip design processes, particularly in the area of custom processor design. Custom chips can optimize performance and energy efficiency for specific applications, but traditional chip design processes are often lengthy and costly, making it difficult to adapt to rapidly changing market needs.

This thesis focuses on chip design under the RISC-V instruction set architecture, aiming to explore the application of current agile development tools and the potential for applying agile development methods in education. By analyzing current trends and cost issues in the chip industry, the thesis proposes strategies for optimizing the chip design process using the open-source RISC-V architecture and agile development methods. Several tools that enhance chip development and debugging, including simulators, differential testing, and debugging components, were developed and adapted. The development process is validated through the design of a four-stage pipeline RISC-V chip. The results show that the RISC-V chip developed works properly, and the development process helps to speed up chip iteration and debugging. Future research will focus on optimizing processor performance assessment and FPGA implementation verification to support broader application needs.