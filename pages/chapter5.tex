\chapter{总结与展望}

\section{本文工作}

本文的主要贡献可以归结为以下两个方面:首先,本研究提出了一套针对芯片前期开发的可复用流程。该流程涵盖了从程序编译到仿真测试再到迭代优化的全过程,极大地提高了开发效率并减少了开发周期。其次,依托于该开发流程,本文使用Chisel语言设计并实现了一个基于四级流水线架构的RISC-V处理器。该处理器不仅支持基本的运算,还实现了简单的异常处理机制和总线接口,为处理更复杂的运算任务和系统扩展提供了可能。

为了提高芯片开发的效率和准确性,本研究基于NEMU模拟器和difftest差分测试工具,构建了一个可调试的芯片开发测试环境。该环境不仅支持详细的执行追踪和断点设置,还能实现自动化的错误检测,显著提高了开发流程的可靠性和工作效率。

进一步地,本文对abstract-machine进行了重要的改进。Abstract-machine原本是一个高度集成的硬件抽象层,用于简化硬件操作的复杂性。本研究对其进行了解耦和重构,将其打包为独立的软件库,便于第三方软件调用和集成。这一改动使得abstract-machine不仅在本项目中发挥了重要作用,也为其他研究者和开发者提供了一个可重用的、高效的硬件抽象接口。

另外,本文设计了一个四级流水线的处理器,详细介绍了处理器的每一个组成部分,包括指令解码、执行单元的设计以及流水线的控制逻辑。同时,本文还针对流水线设计中遇到的各种冒险进行了分析,并提出了相应的解决策略,为类似的芯片设计提供一定的参考。

\section{未来工作}

\subsection{处理器性能优化}

尽管已经实现了基本的流水线架构,但当前处理器的性能仍有待提升。未来工作将重点关注以下几个方面:

\begin{itemize}
    \item 通过引入指令缓存(I-cache)和数据缓存(D-cache),可以显著减少处理器访问主存储器的次数,从而减少延迟并提高指令执行速率。本文中处理器的取指和执行单元都要通过总线访问内存,假如内存的访问周期大于1时,流水线大多数时间都会处于阻塞状态。
    \item 增加转发单元,使其和计分板一同工作,从而减少非必要的数据冒险。通过设计高效的转发单元,可以直接从执行单元将数据转发到需要它的流水线阶段,而不必等待数据写回寄存器文件。
    \item 增加分支预测单元。分支指令的预测不准确会导致流水线的效率大幅下降,因为错误的预测需要撤销后续所有错误路径上的指令。通过引入一个高效的分支预测单元,可以提前正确预测程序的执行路径,从而减少因分支错误导致的处理器资源浪费。本文中固定取下一条指令的策略在遇到向前跳转的循环时,预测准确率非常低。
\end{itemize}

\subsection{FPGA验证}

目前处理器实现了AXI4-Lite总线协议的支持,未来将计划将其部署在FPGA上进行实际的硬件验证。这一步骤是验证处理器设计正确性及其性能的关键。通过FPGA验证,可以实际观察处理器在真实硬件条件下的运行状态和性能表现,及时发现设计上的不足,并对处理器设计进行必要的调整。此外,FPGA平台的使用还将为处理器与外部设备的交互提供实验环境,进一步验证系统的稳定性和可靠性。
