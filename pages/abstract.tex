% 中文摘要
在当前全球科技迅速发展的背景下,物联网、人工智能等前沿技术对高效能、低能耗且能快速构建的芯片提出了迫切需求。这种需求推动了对芯片设计流程的重大创新,特别是在自定义处理器的设计领域。自定义芯片可以针对特定应用优化其性能和能效,但传统的芯片设计流程往往长且成本高昂,难以适应市场快速变化的需求。

本论文聚焦于RISC-V指令集架构下的芯片设计,旨在探索当前敏捷开发工具的应用方式,以及将敏捷开发方法应用于教育的可能性。论文通过分析当前芯片行业的发展趋势和成本问题,提出了采用开源RISC-V架构和敏捷开发方法以优化芯片设计流程的策略。本论文开发、适配了多个能够提高芯片开发、调试的工具,包括模拟器、差分测试和调试组件,并通过设计一个四级流水的RISC-V芯片,对开发流程进行了验证。结果表明,文中开发的RISC-V芯片能够正常工作,并且开发流程有助于提升芯片迭代和调试的速度。未来研究将关注于处理器性能的优化评估和FPGA实现验证,以支持更广泛的应用需求。
