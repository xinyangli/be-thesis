% Chapter 1

\chapter{绪论}

\section{课题研究背景及意义}

% 文献:将定制芯片作为边缘计算解决方案
随着科技的飞速发展,芯片行业已成为全球技术创新的核心领域。由于芯片制程的发展逐渐放缓以及先进制程芯片的成本居高不下,针对边缘计算和加速任务的需求日益增加,定制化芯片设计成为了优先的技术解决方案。在这样的背景下,作为一种开放且免版税的指令集架构,RISC-V凭借其出色的灵活性和可扩展性,已迅速成为芯片设计的重要选择。这种指令集架构为研究人员和商业公司提供了巨大的设计自由度,使芯片设计者能够根据应用需求定制指令集,针对从低功耗物联网设备到高性能计算应用,优化芯片的性能和能效。

% 文献:列举一些应用敏捷开发流程的例子
传统的芯片开发设计周期通常长达3-5年,这在快速变化的市场需求和高压的市场环境下显得不够灵活。长周期的开发时间意味着一旦芯片问世,其设计可能已无法完全适应市场的定制需求。因此,近年来,许多研究工作致力于将软件中的敏捷开发方法引入硬件设计和开发领域\cite{bachrachChiselConstructingHardware2012,xuDevelopingHighPerformance2022}。实践证明,敏捷开发流程的应用能够显著提升中小规模团队的芯片生产效率以及降低开发成本。

% 开放授权指令集带来的开源项目优势
另一方面,RISC-V指令集的开放性极大地促进了学术界与工业界对开源社区的贡献,因为这允许他们在避免高昂的成本和潜在的法律风险的情况下推动创新和开发。开放性的授权促成了一个活跃的开源社区的迅速形成和持续发展。开源RISC-V核心、工具链和模拟工具被大量开发并分享出来,社区利用这些工具提升开发效率,并最终反哺整个RISC-V生态。

近年来,我国一些高等院校开始创新性地应用开源工具于教学之中\cite{YuanJiSuanJiXiTongDaoLunKeChengJiaoXueSiLuJiKeChengZiYuanJianShe2023},用以替代传统课程中常用的Vivado和Quartus进行芯片设计。尽管如此,由于相关开源工具链的中文文档较少且易用性有限,这限制了其在教学中的更广泛应用。

\section{国内外研究现状}

\subsection{RISC-V指令集}

RISC-V指令集由加州大学伯克利分校的研究者在2011年提出\cite{watermanRISCVInstructionSet2011}。作为一种开源的指令集架构(ISA),自其诞生以来已经在全球范围内引起了广泛的关注和研究。由于其开放性和可扩展性,RISC-V不仅吸引了学术界的关注,它还被很多商业公司视为未来的发展方向。

在全球范围内,许多知名的研究机构和企业已经开始探索RISC-V指令集的各种应用。在物联网领域,RISC-V的市场占有率已经达到了28\%;尽管在通用计算领域还没有成熟的RISC-V芯片产品,但是学界和工业界都在积极地探索和尝试,目前已经有一些CPU产品的性能指标接近其他指令集的处理器,如SiFive P650。

在中国,RISC-V也受到了很大的关注。由于芯片制裁等原因,用X86和ARM指令集开发芯片可能会给国内研究者和厂商带来位置的风险。相比之下,RISC-V的开源特性受到了很多新项目的青睐。如中科院计算所牵头的“香山”高性能开源 RISC-V 处理器项目,阿里“玄铁”处理器和一些个人设计的RISC-V芯片\cite{XieJiYuRISCDeSiFaSheChaoBiaoLiangChuLiQiGuanJianJiaGouSheJi2024,JiaRISCVChuLiQiHeSheJiYouHuaYuKuoZhanZhiLingJiShiXianYanJiu2023}。

未来,RISC-V可能会在多个方向上发展。随着技术的成熟,RISC-V可能会在更多高性能计算和大数据处理领域得到应用。同时,随着更多的开源社区和企业的加入,RISC-V的生态系统将变得更加丰富和强大。此外,RISC-V架构的灵活性使其成为在新兴技术领域,如人工智能和机器学习硬件开发中的一个有吸引力的选项\cite{newsRISCVFutureMachine}。

\subsection{敏捷开发在芯片设计中的应用}

% 插图:芯片开发流程
敏捷开发方法在软件领域已经被广泛应用,极大提高了软件项目的适应性、效率和质量。不过不同于软件,由于成本、开发周期等原因,传统的芯片开发流程基本遵循“瀑布式开发”的原则,即在工期内仅进行一次迭代,设计、验证、流片等环节需要等待上一个环节开发完成。

随着摩尔定律逐渐失效,通用计算领域的算力增长逐渐放缓,而通过专业芯片来进行计算加速成为了很有前景的方向。一些研究机构开始尝试将敏捷开发方法应用于硬件开发过程\cite{leeAgileApproachBuilding2016,baoAgileOpenSourceHardware2020},希望以此来加快专用领域芯片开发。

最早,敏捷开发主要被用于前期的原型验证,即前端设计和仿真上。利用Chisel语言\cite{bachrachChiselConstructingHardware2012}参数化和函数式编程的功能,生成自定义的IP核,使得SoC的开发变得更加快速。例如,RocketChip\cite{ChipsallianceRocketchip2023}和BOOM\cite{zhaoSonicBOOM3rdGeneration}都支持通过修改参数来定制CPU核心;Chipyard\cite{amidChipyardIntegratedDesign2020} 则在它们的基础上,为定制SoC提供了解决方案。

目前也有一些工作尝试构建新的仿真、综合、验证工具,来将芯片设计中更多环节敏捷化。例如,LiveHD\cite{wangLiveHDProductiveLive2020a}探讨了如何通过增量综合的方式,来大幅减少仅做少量修改时的综合时间;Verilator和ESSENT通过提高仿真速度\cite{beamerCaseAcceleratingSoftware2020},来缩短芯片初期开发时的反馈周期。

不过目前工业界对于敏捷开发的应用还较少,尤其是在仿真验证和后端设计环节。在门级仿真方面,工业界常使用的工具包括ModelSim、VCS以及Cadence NCSim。此外,对于后端设计,常见的商业化工具有Cadence Encounter、Synopsys IC Compiler和Mentor Graphics Calibre。这些工具都有成熟的应用和可靠性保证,而芯片行业的试错成本很高,商业公司不愿意为此承担风险。现有的芯片设计工程师已经熟悉了原有开发流程,推动新工具应用的意愿较低。

因此,开源芯片和小团队芯片开发仍然是硬件敏捷开发主要的应用领域。敏捷开发的引入缩短了芯片设计到验证之间的反馈周期,便于芯片功能的迭代升级和问题修复。例如,XiangShan\cite{xuDevelopingHighPerformance2022} 处理器引入了敏捷开发流程,从建立代码仓库到Debian正常启动只用了4个月的时间。

\subsection{流水线处理器设计}

流水线处理器设计是一种使处理器在同一时刻可以执行多个操作的技术。通过将指令的执行分解为多个独立的步骤,每个步骤在处理器的不同部分并行进行,流水线极大地提高了处理器的效率和吞吐率。

在现代处理器架构中,根据处理能力,处理器可以分为标量处理器和超标量处理器。标量处理器每个时钟周期执行一个操作,而超标量处理器可以在每个时钟周期执行多个操作。这种分类基于处理器的执行能力,即其每个时钟周期内能处理的指令数。

处理器的执行方式可以进一步分为顺序执行和乱序执行。顺序执行要求按照程序的指令顺序严格执行,而乱序执行则允许指令根据资源的可用性和依赖关系在任意顺序执行。乱序执行可以优化指令的执行效率和处理器资源的利用,减少因等待必要数据或执行单元而造成的空闲时间。

设计高效的流水线是处理器设计中的一项关键任务,涉及到流水线的深度和各级的平衡。流水线的每个级别都应当尽可能均衡,避免某一级成为瓶颈。例如,如果某一级的处理时间过长,则它将限制整个流水线的性能,因为其他级别必须等待这一级完成。

流水线设计还面临如何处理各种冒险的挑战。冒险主要有三种:结构冒险、数据冒险和控制冒险。
\begin{itemize}
    \item 结构冒险:发生在硬件资源不足以支持所有并发操作时。
    \item 数据冒险:发生在后续指令依赖于前一指令的结果时。
    \item 控制冒险:涉及到分支指令,如条件跳转可能导致的执行流变更。
\end{itemize}

为了减少这些冒险对性能的影响,现代处理器采用了多种高级技术,如指令重排、分支预测和动态执行,这些技术可以优化指令流并提高流水线的效率。

流水线处理器设计不仅仅是提高处理速度,它还需要智能地管理和优化处理器内部的资源和执行过程。通过持续的技术创新和优化,现代处理器能够在高效处理大量数据的同时,最大限度地降低延迟和提高计算吞吐率。

\subsection{RISC-V软件模拟器}

模拟器是用于在软件环境中重现和测试硬件设备的行为的工具,而仿真器主要用于从HDL代码入手,模拟和验证电子电路设计的行为和逻辑,它们能确保芯片能够正确地满足功能和性能要求。
现有RISC-V指令集下的方案主要分为几类:指令层面的模拟器,如Spike\cite{RiscvsoftwaresrcRiscvisasim2023}和QEMU\cite{bellardQEMUFastPortable2005};RTL层面的模拟器,即利用运行在仿真器上的RocketChip\cite{ChipsallianceRocketchip2023}和BOOM\cite{zhaoSonicBOOM3rdGeneration} 来作为模拟器;系统级处理器模拟器,如Gem5\cite{lowe-powerGem5SimulatorVersion2020,roelkeRISC5ImplementingRISCV2017}。

在应用于辅助硬件开发时,不同方案有其优缺点。指令层面的模拟器执行速度快,可以快速得到指令运行的结果进行对比;但是其很难模拟真实系统上的时延,在出现问题时,也不能提供微架构层面的信息。RTL层面的模拟能够最真实的反应系统的状态;但是对于复杂的程序,其执行速度过慢。Gem5定位在这两种模拟器之间,采用事件驱动的模拟方式,通过详细的配置,可以真实的反映真实系统上的时间;不过其配置复杂,对于初期的CPU设计帮助不大。

\section{论文的主要内容与组织架构}

本文全程采用开源工具进行CPU设计和仿真,对设计过程中用到的开源工具进行了修改、打包和封装,以提高其易用性。研究成功基于这些工具开发了支持RV32E指令集的软件模拟器和CPU,这一成果展示了开源工具的实用性,为其在高等教育课程中的广泛推广和应用提供一定的参考。

本文内容组织如下:

第一章为绪论,首先介绍了选择该研究主题的背景及其科学与实际应用的重要性。接着,回顾和评述了国内外在敏捷开发在芯片设计中的应用、流水线处理器设计、以及片上系统连接等方面的研究现状和进展,为进一步的研究提供了理论依据和技术支持。

第二章着重于CPU的设计。介绍了基于消息控制的处理器设计理念,强调其在教学和实际应用中的优势,例如提高可扩展性和灵活性。详细描述了处理器设计的各个阶段,包括取指阶段、译码阶段、执行阶段和写回阶段的实现。讨论了如何处理流水线设计中遇到的结构冒险、数据冒险和控制冒险。

第三章介绍仿真验证平台的搭建,解释了差分测试在验证芯片设计中的作用和重要性,以及如何实施差分测试以提高芯片仿真的效率和准确性。讨论了软件模拟器的设计和实现,包括指令解释器、基于内存映射的设备模拟和异常处理的详细实现。

第四章对处理器进行仿真验证,描述了用于测试和验证处理器设计的测试程序,以及运行这些程序得到的结果。探讨了如何建立一个持续的测试平台来持续验证和优化处理器设计。

第五章为总结与展望,总结了全文的研究成果和论文的主要贡献,并展望了未来研究的方向,包括处理器性能的进一步优化和 FPGA 验证。



